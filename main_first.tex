% latexmk -dvi main_first.tex
% dvips main_first.dvi
% ps2pdf -dALLOWPSTRANSPARENCY main_first.ps

\documentclass{article}
\usepackage{float,paracol,gensymb,amssymb,graphicx,calculator,pstricks,pst-plot,pst-math}
\usepackage[margin=1in,a4paper,landscape]{geometry}
\usepackage[hidelinks]{hyperref}
\hypersetup{linktoc=all}

\title{Pre-Calculus}
\author{\emph{Content}\\Sarah Eichorn \qquad Rachel Lehman\\University of California, Irvine
\and \emph{Document}\\Tim Gudlewski, University at Albany, SUNY}
\date{2021}

\begin{document}
\maketitle
\setcounter{secnumdepth}{2}
\setcounter{tocdepth}{2}
\tableofcontents
\newpage
\section{Trigonometry}
\newpage

\begin{paracol}{2}
\subsection{Coterminal Angles}
\subsubsection{Example 1: Find an angle between $0$ and $2\pi$ that is coterminal with
$\theta = -\frac{11\pi}{6}$.}
\begin{itemize}
  \item Two angles are coterminal if their starting and ending sides are the same.
  \item An angle is in standard position when its initial side is on the positive $x$-axis.
  \item For negative angles, rotate clockwise.
  \item $-2\pi = -\frac{12\pi}{6}$, therefore we're $\frac{\pi}{6}$ short.
  \item $-\frac{11\pi}{6}+2\pi = \frac{\pi}{6}$
  \item \textbf{Answer:} The coterminal angle is $\frac{\pi}{6}$
\end{itemize}
\switchcolumn
\begin{figure}[H]
  \centering
  \begin{pspicture}(-2.5,-2.5)(2.5,2.5)
    \psaxes[ticks=none, labels=none]{<->}(0,0)(-2.5,-2.5)(2.5,2.5)
    \psarcn{->}(0,0){1}{0}{!PI 1.8333 mul neg RadToDeg}  % 11pi/6
    \psarcn{->}(0,0){2}{0}{!PI 2 mul neg RadToDeg}  % 2pi
    \SpecialCoor
    \uput{0.05}[45](1;30){$-\frac{11\pi}{6}$}
    \uput[45](2,0){$-2\pi$}
\end{pspicture}
  \caption{Coterminal angles in radians}
\end{figure}
\switchcolumn*

\subsubsection{Example 2: Let $\theta = 600\degree$.\\a) Find an angle between $0\degree$ and $360\degree$
that is coterminal with $\theta$.\\b) Find an angle between $0\degree$ and $-360\degree$
that is coterminal with $\theta$.}
\begin{itemize}
  \item $360\degree+180\degree = 540\degree$
  \item $540\degree+60\degree = 600\degree$
  \item $180\degree+60\degree = 240\degree$
  \item \textbf{Answer a):} $240\degree$
  \item $180\degree-60\degree = 120\degree$
  \item \textbf{Answer b):} $120\degree$
\end{itemize}\switchcolumn
\begin{figure}[H]
  \centering
  \begin{pspicture}(-3,-3)(3,3)
  \psaxes[ticks=none,labels=none]{<->}(0,0)(-3,-3)(3,3)
  \psarc[linecolor=red]{->}(0,0){1}{0}{360}
  \psarc[linecolor=red]{->}(0,0){1.5}{360}{540}
  \psarc[linecolor=red]{->}(0,0){1.5}{540}{600}
  \psarc[linecolor=blue,linestyle=dashed]{->}(0,0){2.5}{0}{180}
  \psarc[linecolor=blue,linestyle=dashed]{->}(0,0){2.5}{180}{240}
  \psarcn[linecolor=green,linestyle=dotted]{->}(0,0){2.5}{0}{-120}
  \SpecialCoor
  \psline[linecolor=red]{->}(3;0)
  \psline[linecolor=red]{->}(3;600)
  \psline[linestyle=dashed,linecolor=blue]{->}(3;0)
  \psline[linestyle=dashed,linecolor=blue]{->}(3;240)
  \psline[linestyle=dotted,linecolor=green]{->}(3;0)
  \psline[linestyle=dotted,linecolor=green]{->}(3;-120)
  \uput{0.1}[225](1,0){\color{red}$360\degree$}
  \uput{0.1}[130](-1.5,0){\color{red}$540\degree$}
  \uput{0.1}[130](-2.5,0){\color{blue}$180\degree$}
  \uput{0.25}[180](1.5;240){\color{red}$600\degree$}
  \uput{0.25}[180](2.5;240){\color{blue}$240\degree$}
  \uput[-60](2.5;-45){\color{green}$-120\degree$}
\end{pspicture}

  \caption{Coterminal angles in degrees}
\end{figure}
\switchcolumn
\flushpage

\subsection{Reference Angles}
\subsubsection{Example 1: Find the reference angle for $\theta = \frac{5\pi}{6}$.}
\begin{itemize}
  \item The reference angle is the acute positive angle formed by the terminal
  side of $\theta$ and the $x$-axis.
  \item Reference angle = $\pi-\frac{5\pi}{6} = \frac{\pi}{6}$
  \item \textbf{Answer:} $\frac{\pi}{6}$
\end{itemize}\switchcolumn
\begin{figure}[H]
  \centering
  \begin{pspicture}(-2.5,-2.5)(2.5,2.5)
  \psaxes[ticks=none,labels=none]{<->}(0,0)(-2.5,-2.5)(2.5,2.5)[$x$,90][$y$,0]
  \psarc[linecolor=red]{->}(0,0){1}{0}{!PI 0.8333 mul RadToDeg}  % 5pi/6
  \psarc[linecolor=blue,linestyle=dashed](0,0){1}{!PI 0.8333 mul RadToDeg}{!PI RadToDeg}
  \SpecialCoor
  \psline[linecolor=red]{->}(2.5;150)
  \psline[linecolor=red]{->}(2.5;0)
  \psline[linecolor=blue,linestyle=dashed]{->}(2.5;150)
  \psline[linecolor=blue,linestyle=dashed]{->}(2.5;180)
  \uput[45](1;45){\color{red}$\frac{5\pi}{6}$}
  \uput[165](1;165){\color{blue}(ref. angle) $\frac{\pi}{6}$}
\end{pspicture}

  \caption{Reference angle in radians}
\end{figure}\switchcolumn*
\subsubsection{Example 2: Find the reference angle for $\theta = -156\degree$.}
\begin{itemize}
  \item Reference angle = $180\degree-156\degree = 24\degree$
  \item \textbf{Answer:} $24\degree$
\end{itemize}
\switchcolumn
\begin{figure}[H]
  \centering
  \begin{pspicture}(-2.5,-2.5)(2.5,2.5)
  \psaxes[ticks=none,labels=none]{<->}(0,0)(-2.5,-2.5)(2.5,2.5)[$x$,90][$y$,0]
  \psarcn[linecolor=red]{->}(0,0){1}{0}{-156}
  \SUBTRACT{180}{156}{\tempA}
  \ADD{180}{\tempA}{\tempB}
  \psarc[linecolor=blue,linestyle=dashed](0,0){1}{180}{\tempB}
  \SpecialCoor
  \psline[linecolor=red]{->}(2.5;-156)
  \psline[linecolor=red]{->}(2.5;0)
  \psline[linecolor=blue,linestyle=dashed]{->}(2.5;\tempB)
  \psline[linecolor=blue,linestyle=dashed]{->}(2.5;180)
  \uput[-60](1;-60){\color{red}$-156\degree$}
  \uput[192](1;192){\color{blue}(ref. angle) $\tempA\degree$}
\end{pspicture}

  \caption{Reference angle in degrees}
\end{figure}
\switchcolumn
\flushpage
\subsubsection{Example 3: Find the reference angle for $\theta = \frac{17\pi}{9}$.}
\begin{itemize}
  \item Reference angle = $2\pi-\frac{17\pi}{9} = \frac{\pi}{9}$
  \item \textbf{Answer:} $\frac{\pi}{9}$
\end{itemize}
\switchcolumn
\begin{figure}[H]
  \centering
  \begin{pspicture}(-2.5,-2.5)(2.5,2.5)
  \psaxes[ticks=none,labels=none]{<->}(0,0)(-2.5,-2.5)(2.5,2.5)[$x$,90][$y$,0]
  \psarc[linecolor=red]{->}(0,0){1}{0}{!PI 1.8888 mul RadToDeg}  % 17pi/9
  \psarcn[linecolor=blue,linestyle=dashed](0,0){1}{0}{!PI 0.1111 mul neg RadToDeg}
  \SpecialCoor
  \psline[linecolor=red]{->}(2.5;0)
  \psline[linecolor=red]{->}(2.5;340)
  \psline[linecolor=blue,linestyle=dashed]{->}(2.5;340)
  \psline[linecolor=blue,linestyle=dashed]{->}(2.5;360)
  \uput[225](1;225){\color{red}$\frac{17\pi}{9}$}
  \uput{0.5}[350](1;350){\color{blue}$\frac{\pi}{9}$ (ref. angle)}
\end{pspicture}

  \caption{Reference angle in radians}
\end{figure}
\switchcolumn
\flushpage

\subsection{Sketching an Angle in Standard Position}
\subsubsection{Example 1: Sketch $\theta = \frac{2\pi}{3}$ in standard position.}
\begin{itemize}
  \item An angle is in standard position if its initial side is on the positive $x$-axis.
  \item If $\theta>0$, rotate counter-clockwise. If $\theta<0$, rotate clockwise.
  \item One complete rotation measures $2\pi$ radians.
  \begin{itemize}
    \item The radian measure of an angle is defined as the length of the corresponding arc on the unit circle.
    \item Since the circumference of the unit circle is $2\pi$, then the radian measure of one complete revolution is $2\pi$.
    \item $\therefore$ half a rotation would measure $\pi$ radians.
  \end{itemize}
  \item If $\pi$ is split into thirds, $\frac{2\pi}{3}$ would span two of them.
\end{itemize}
\switchcolumn
\begin{figure}[H]
  \centering
  \begin{pspicture}(-2.5,-2.5)(2.5,2.5)
  \psaxes[ticks=none,labels=none]{<->}(0,0)(-2.5,-2.5)(2.5,2.5)[$x$,90][$y$,0]
  \psarc[linecolor=red]{->}(0,0){1}{0}{!PI 0.6666 mul RadToDeg}  % 2pi/3
  \psarc[linecolor=green,linestyle=dotted]{->}(0,0){1}{0}{!PI 0.3333 mul RadToDeg}  % pi/3
  \psarc[linecolor=blue,linestyle=dashed]{->}(0,0){2}{0}{!PI RadToDeg}
  \psarc[linecolor=purple,linestyle=dotted]{->}(0,0){2}{0}{!PI 2 mul RadToDeg}
  \SpecialCoor
  \psline[linecolor=red]{->}(2.5;0)
  \psline[linecolor=red]{->}(2.5;120)
  \psline[linecolor=blue,linestyle=dashed]{->}(2.5;0)
  \psline[linecolor=blue,linestyle=dashed]{->}(2.5;180)
  \psline[linecolor=purple,linestyle=dotted]{->}(2.5;0)
  \psline[linecolor=purple,linestyle=dotted]{->}(2.5;360)
  \psline[linecolor=green,linestyle=dotted]{->}(2.5;0)
  \psline[linecolor=green,linestyle=dotted]{->}(2.5;60)
  \uput[105](1;105){\color{red}$\frac{2\pi}{3}$}
  \uput[45](1;45){\color{green}$\frac{\pi}{3}$}
  \uput[135](2;180){\color{blue}$\pi$}
  \uput[-45](2;0){\color{purple}$2\pi$}
\end{pspicture}

  \caption{$\theta$ in standard position}
\end{figure}\switchcolumn*
\subsubsection{Example 2: Sketch $\theta = -\frac{5\pi}{4}$ in standard position.}
\begin{itemize}
  \item If $-2\pi$ is split into fourths, $-\frac{5\pi}{4}$ would span five of them going clockwise.
\end{itemize}
\switchcolumn
\begin{figure}[H]
  \centering
  \begin{pspicture}(-2.5,-2.5)(2.5,2.5)
  \psaxes[ticks=none,labels=none]{<->}(0,0)(-2.5,-2.5)(2.5,2.5)[$x$,90][$y$,0]
  \psarcn[linecolor=red]{->}(0,0){1}{0}{!PI 1.25 mul neg RadToDeg}  % -5pi/4
  \psline[linecolor=red]{->}(2.5;0)
  \psline[linecolor=red]{->}(2.5;-225)
  \uput[-200](1;-200){\color{red}$-\frac{5\pi}{4}$}
\end{pspicture}

  \caption{$\theta$ in standard position (negative)}
\end{figure}






\end{paracol}
\end{document}
